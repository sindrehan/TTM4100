% Created 2016-04-22 fr. 14:52
\documentclass[11pt]{article}
\usepackage[utf8]{inputenc}
\usepackage[T1]{fontenc}
\usepackage{fixltx2e}
\usepackage{graphicx}
\usepackage{longtable}
\usepackage{float}
\usepackage{wrapfig}
\usepackage{rotating}
\usepackage[normalem]{ulem}
\usepackage{amsmath}
\usepackage{textcomp}
\usepackage{marvosym}
\usepackage{wasysym}
\usepackage{amssymb}
\usepackage{hyperref}
\tolerance=1000
\author{Sindre Hansen}
\date{\today}
\title{Exercise 8}
\hypersetup{
  pdfkeywords={},
  pdfsubject={},
  pdfcreator={Emacs 24.5.1 (Org mode 8.2.10)}}
\begin{document}

\maketitle

\section{Task 1}
\label{sec-1}
\begin{center}
\begin{tabular}{lll}
Service & Corresponding in IP & Corresponding in TCP\\
\hline
Full duplex & NO & YES\\
Flow controls & NO & YES\\
Error detection & YES & YES\\
Error correction & NO & NO\\
Framing & YES & YES\\
Link access & NO & NO\\
Reliable delivery & NO & YES\\
\end{tabular}
\end{center}

\section{Task 2}
\label{sec-2}
If d$_{\text{drop}}$ < L/R then there will be a collision. The reason being that before one node finishes transmitting, it will start receiving from the other node.
\section{Task 3}
\label{sec-3}
A wireless network is either in infrastructure mode or in ad-hoc mode. In infrastructure mode every host is connected to the network via a access point. In ad-hoc mode there's no such access point and every host provides their own routing, name translation etc.
\section{Task 4}
\label{sec-4}
If a node attaches to the network in different points over time, it is defined as a mobile node. Thus, a node with a wireless connection is \textbf{not} necessarily mobile. The user is \textbf{not} mobile, since she is continuously connected to the internet through the same access point.

\section{Task 5}
\label{sec-5}
d$_{\text{0}}$ = [[1,-1,1,-1,1,-1,1,-1]\\
d$_{\text{1}}$ = [-1,1,-1,1,-1,1,-1,1]
% Emacs 24.5.1 (Org mode 8.2.10)
\end{document}
